\chapter{Descrição da \textsf{classe-ifg}}
\label{cap:descr}

%% - - - - - - - - - - - - - - - - - - - - - - - - - - - - - - - - - - -
\section{Opções da classe}
\label{sec:opcoes}
Para usar esta classe num documento \LaTeXe, coloque a pasta \verb|formatacao| numa pasta onde o compilador \LaTeX\ pode achá--lo (normalmente na mesma pasta que seu arquivo \verb|.tex|), e defina--o como o estilo do seu documento. Por exemplo, uma dissertação de mestrado:
\begin{verbatim}
\documentclass[dissertacao]{classe-ifg}
...
\begin{document}
\end{verbatim}

As opções da classe são \verb|[dissertacao]| (para dissertação de mestrado), \verb|[qualificacaom]| (para qualificação de mestrado) e \verb|[monografia]| (para monografia final de especialização). Se nenhuma opção for declarada, o documento é considerado como uma dissertação de mestrado. %Com a opção \verb|[nocolorlinks]| todos os {\em links} de navegação no texto ficam na cor preta. O ideal é usar esta opção para gerar o arquivo para impressão, pois a qualidade da impressão dos {\em links} fica superior.


%% - - - - - - - - - - - - - - - - - - - - - - - - - - - - - - - - - - -
\section{Parâmetros da classe}
\label{sec:param}
Os elementos pré-textuais são definidos página por página e dependem da correta definição dos parâmetros listados a seguir (os elementos que não foram aplicáveis como, por exemplo, \verb|\orientadora| quando orientador é do sexo masculino, devem permanecer comentados usando \% no início da respectiva linha):

% \setlength{\topsep}{5.2em}
 \begin{itemize}%\addtolength{\itemsep}{-0.7em}
\item \verb|\autor| : Nome completo do autor, começando pelo primeiro nome (ex.: José da Silva);
\item \verb|\autorR| : Nome completo do autor, começando pelo último sobrenome (ex.: da Silva, José);
\item \verb|\sautor| : Nome completo do segundo autor (quando aplicável), começando pelo primeiro nome (ex.: José da Silva);
\item \verb|\sautorR| : Nome completo do segundo autor (quando aplicável), começando pelo último sobrenome (ex.: da Silva, José);
\item \verb|\titulo| : Título da dissertação ou monografia de conclusão de curso;
\item \verb|\subtitulo| : Se tiver um subtítulo, use este macro para defini--lo;

\item \verb|\cidade| : A cidade de edição. 
\item \verb|\dia| : Dia do mês da data de defesa (01--31);
\item \verb|\mes| : Mês da data de defesa (01--12);
\item \verb|\ano| : Ano da data de defesa;

\item \verb|\orientador| : Nome completo do orientador, começando pelo primeiro nome;
\item \verb|\orientadorR| : Nome completo do orientador, começando pelo último sobrenome;

\item \verb|\orientadora| : Nome completo da orientadora, começando pelo primeiro nome; use este comando e o próximo se for orientadora e não orientador.
\item \verb|\orientadoraR| : Nome completo do orientadora, começando pelo último sobrenome;

\item \verb|\coorientador| : Nome completo do co--orientador, começando pelo primeiro nome;
\item \verb|\coorientadorR| : Nome completo do co--orientador, começando pelo último sobrenome;

\item \verb|\coorientadora| : Nome completo da coorientadora, começando pelo primeiro nome; use este comando e o próximo se for coorientadora e não coorientador.
\item \verb|\coorientadoraR| : Nome completo do coorientadora, começando pelo último sobrenome;

\item \verb|\universidadeco| : Nome da universidade do coorientador;
\item \verb|\unico| : Sigla da universidade do coorientador;
\item \verb|\unidadeco| : Nome da unidade acadêmica do coorientador.\footnote{Se não tiver um co--orientador, não defina esses últimos sete parâmetros.}
\end{itemize}

%% - - - - - - - - - - - - - - - - - - - - - - - - - - - - - - - - - - -
\section{Elementos Pré--Textuais}
\label{sec:pre}
Os elementos pré--textuais são definidos página por página, conforme descritos a seguir:

\paragraph{capa\\}
\verb|\capa| : Gera o modelo da capa externa do trabalho. Esta página servirá apenas como modelo para a encadernação da versão final do texto. Nenhum dado é necessário.

\paragraph{rosto\\}
\verb|\rosto| : Gera a folha de rosto, a qual é a primeira folha interna do trabalho. Nenhum dado é necessário.

\paragraph{ficha\\}
\verb|\ficha| : Inclui a ficha catalográfica que deve constar como um arquivo PDF cujo nome deve necessariamente ser \verb|ficha.pdf|. Este arquivo deve estar localizado na mesma pasta que seu arquivo \verb|.tex|. Comente este comando até ter a versão final da dissertação.

\paragraph{aprovacao\\}
\verb|\input{./pre/preAprovacao}| : Entrada para o nome dos examinadores, exceto o(s) orientador(es). Edite o arquivo equivalente com os parâmetros nele descritos.

%\paragraph{curriculo\\}
%\verb|\aprovacao| : ambiente para a reprodução do termo de aprovação da Banca Examinadora da tese ou dissertação.
%

\paragraph{dedicatória\\}
\verb|\input{./pre/preDedicatoria| : ambiente para escrever a dedicatória. É possível trocar o espaçamento dentro desse ambiente do mesmo jeito que no \LaTeX\ padrão.

\paragraph{agradecimentos\\}
\verb|\begin{agradecimentos}

Texto de agradecimento.

\end{agradecimentos}| : ambiente para escrever os agradecimentos. É possível trocar o espaçamento dentro desse ambiente do mesmo jeito que no \LaTeX\ padrão.

\paragraph{resumo\\}
\verb|\chaves{palavra 1, palavra 2, palavra 3} % Palavras chave separadas por vírgula

\begin{resumo}

De 150 a 500 palavras - trabalhos acadêmicos (teses, dissertações e outros) e relatórios técnico-científicos (ABNT NBR 6028)

\end{resumo}
| : A lista das palavras chaves, separadas por `;'. Deve ser definido antes do ambiente \verb|\resumo|, o qual é usado para escrever o resumo em português.

\paragraph{abstract\\}
\verb|\keys| : A lista das palavras chaves em inglês, separadas por `;'. Deve ser definido antes do ambiente \verb|\abstract|, o qual contém 1 argumento e é usado escrever o resumo em inglês. O argumento deve ser o título do trabalho em inglês.

\paragraph{tabelas\\}
\verb|\tabelas| : Macro com 1 argumento opcional para gerar as tabelas. O argumento pode ser:
\begin{itemize}
 \item nada [] : gera apenas o sumário;
 \item \textsf{fig} : gera o sumário e uma lista de figuras;
 \item \textsf{tab} : gera o sumário e uma lista de tabelas;
 \item \textsf{alg} : gera o sumário e uma lista de algoritmos;
 \item \textsf{cod} : gera o sumário e uma lista de programas.
 \item \textsf{figtab} : gera o sumário, uma lista de tabelas, e uma lista de figuras;
 \item \textsf{figtabalg} : gera o sumário e listas de tabelas, de figuras e de algoritmos;
 \item \textsf{figtabalgcod} : gera o sumário e listas de tabelas, de figuras, de algoritmos e de programas;
 \item (qualquer outra coisa) : gera somente o sumário.
\end{itemize}

Pode-se usar qualquer combinação dessas opções. Por exemplo:
\begin{itemize}
 \item \textsf{figtab} : gera o sumário e listas de figuras e tabelas,
 \item \textsf{figtabcod} : gera o sumário e listas de figuras, tabelas e códigos de programas;
 \item \textsf{figtabalg} : gera o sumário e listas de figuras, tabelas e algoritmos;
 \item \textsf{figtabalgcod} : gera o sumário e listas de figuras, tabelas, algoritmos e códigos de programas
\end{itemize}

\paragraph{epígrafe\\}
\verb|\epigrafe| : Macro com 3 argumentos que permite editar um epígrafe. O primeiro argumento é o texto da citação. O segundo argumento é o nome do autor da citação. O terceiro argumento é o título da referência à qual a citação pertence.
